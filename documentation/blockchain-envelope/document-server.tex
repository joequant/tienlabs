\documentclass[journal]{IEEEtran}
\usepackage{verbatim}
\usepackage[pdftex]{graphicx}
\usepackage{url}
\usepackage[backend=bibtex]{biblatex}
\bibliography{document-server}
\begin{document}
\title{Digital Envelope Microservices for Blockchain Document Exchange}
\author{Joseph~Chen-Yu~Wang,~\IEEEmembership{TiEN Foundation}
Willam~Tien,~\IEEEmembership{TiEN Foundation}
\thanks{e-mail: (joe@tienpay.com).}}
\maketitle

DRAFT - 20181201 - INCOMPLETE NOT FOR CITATION - CONTENTS SUBJECT TO CHANGE

\begin{abstract}
Blockchain is a revolutionary technology which has much potential to
improve business processes by allowing for decentralized data
exchange.  However, based on our experiences in attempting to use
blockchain in African coffee trade, we believe that current
applications of blockchain are severely flawed at both the use case
level and implementation level.  In order to address these flaws
we have designed a microservice intended to authenticate and exchange
basic business documents.  This paper outlines the design and
implementation of a digital envelope system, by which arbitrary
documents can be added and annotated into the ethereum blockchain.
\end{abstract}


\begin{IEEEkeywords}
Blockchain
\end{IEEEkeywords}

\section{Introduction}
Blockchain and decentralized finance technology brings with it the
promise of increased efficiency in business processes.  In particular,
blockchain has been cited by the United Nations as a potentially
transformative technology for agriculture.\cite{fao:blockchain} In
addition, blockchain has been identified by President Xi Jinping of
China as a critical technology\cite{xi}, and its application will be an
important part of the China's Belt and Road Initiative.

In order to explore the uses of blockchain, we have attempted to apply
blockchain technology to trading coffee and manufactured goods along
between Hong Kong and Tanzania.  We established a pilot project to
import coffee from Tanzania to Hong Kong, and to also export
manufactured projects from the Greater Bay Area to Tanzania.  In doing
so, we attempted to use a number of blockchain technology.  Among all
of the blockchain technologies we attempted to use, only
cryptocurrencies proved to be useful, and it was our experience that
all other blockchain systems we attempted to use were completely
unusable.

In this paper, we explain why we found blockchain technologies other
than bitcoin proved unsuitable, and we explain current work that we
are undertaking with the TiEN Foundation to address the issues that we
found in our experiences.

\subsection{Importance of Bitcoin}
The only blockchain technology which proved to be useful in our pilot
project was bitcoin.  The use of bitcoin allowed us to instantly
transmit funds between Hong Kong and Tanzania and to track payments.
Our payments consisted of converting Hong Kong dollars to bitcoin via
a cryptocurrency exchange, transmitting it to Tanzania, and then
converting the bitcoin to Tanzanian dollars via a local bitcoin
exchanger.  The use of bitcoin allowed us to transmit payments
instantly and allowed us to track the receipt of payments.  Bitcoin
also enabled us to connect with the local technology community in
Africa.

\subsection{General Limitations of Current Systems}
Aside from the use of bitcoin, we attempted to use blockchain systems
within our coffee trade.  Other than bitcoin, we found that all the
systems we examined were unsuitable for our pilot project. There are a
number of general issues that we found to be problematic in addition
to specific issues associated with agricultural use cases.

Among the issues that we have found in attempting to use blockchain
for business applications are
\begin{itemize}
 \item System integration problem
 \item Fixed workflows problem
 \item Walled gardens problem
 \item Weakest link problem
\end{itemize}

\paragraph {System integration problem} Businesses typically have existing
applications which complicated business logic.  Moreover these systems
must be constantly updated to incorporate bug fixes and enhancements,
and it is impractical to put the entire system into the blockchain.

\paragraph {Fixed workflows problem} Another issue is that because blockchain
smart contracts cannot easily be changed after the contract is pushed
to the blockchain, smart contracts typically assume fixed workflows.
In actual business situations, the workflow is subject to change.
Moreover in situations were there are multiple actors, the entire
workflow may not be known to any particular party.

\paragraph {Walled garden issues problem} The usefulness of blockchain is in
the ability to exchanged data between decentralized actors.  However,
most of the blockchain projects that have been proposed thus far
assume that the actors in the blockchain are from similar industries
within a consortium.  However, this prevents blockchain from being
used to exchange information between different industries.

\paragraph {Weakest link issues problem} One further problem with blockchain
systems is that they typically have weakest leak issues in that there
are no productivity improvements when blockchain is used only for part
of the workflow.

Because of these issues our previous efforts to apply blockchain
technology as a business process have been unsuccessful.  However,
our earlier failures in applying blockchain to coffee supply chains
have given us experience for our current effort.

\subsection{Use Case Limitations}

Aside from the general limitations of blockchain systems, we
identified several specific issues associated with the use cases for
agriculture identified with the United Nations report, namely the use
of blockchain to track land tenure and to assure provenance of goods.

With respect of land tenure, our experience the actual user of the
land is often controlled by informal and highly complex agreements
that are not recorded in the land office.  Rather than recording
formal usage of the land, our experience is that there are often
informal agreements that are not centrally recorded.  Based on our
experience, it is our view that in recording land tenure or other
agreements that instead of a system that records centralized land
transfers, that there would be great usefulness in have a system by
which private agreements between individuals can be recorded and
authenticated for later retrieval.  In addition, even a system that
allows an individual to record a authenticated diary would be useful
in enforcing private informal land use agreements.

With respect to supply chains, we have found that people viewing the
supply chain often view the system as a set of fixed steps which can
then be optimized.  In practice, we found that there was a
coordination problem, while all of the actors in the supply chain may
benefit from increased transparency, each individual actor has no
incentive to adopt new technology and tremendous disincentives to do
so.  In particularly, within the coffee supply chain, each actor
typically is aware of only their upstream source and their downstream
consumer.  Each actor is disincentivized from providing information on
their suppliers and consumers out of fear of being disintermediated.
Also because each actor receives their income from the markup between
producer and consumer, the actors wish to maintain bargaining power by
keeping the prices paid opaque.  In short, while the supply chain as a
whole may benefit from greater transparency, no one actor gains by
making their part of the supply chain more transparent.  To deal with
this issue, we are focusing on technologies that can be adopted by one
link on the chain and which have the potential to create increased
returned through efficiency in one single part of the chain which
would benefit that individual actor.

There are two additional issues which were identified in our pilot
project which relate to vendor lock-in and use of local technical
resources.  In the developed world, there has been a movement to
software-as-a-service models, however it is our experience that the
actors in our supply chain are reluctant to build their business
around a service which may disappear, and this renders software as a
service models problematic.  In addition, we have found that there is
a great deal of local IT talent in Tanzania which can be used to
deploy or customize software particularly open-source systems.
However, there are not the resources available to build and maintain
complex enterprise systems, and the availablity and type of local
software resource leads us away from software as a service models and
toward models involving containerized open source systems.

\section{System Design}

Our unsatisfactory experiences with existing blockchain systems gives
us some insight into systems which we think may be useful for our use
cases.  Rather than creating a general end to end system which runs
into the various problems, we are creating a microservice that
provides simple document processing functionality which can be
incorporated in various business processes.  In addition to the
microservice, we are applying our software to two specific workflows
which can then be used in actual business application.

Our microservice implements the concept of a digital envelope.  The
digital envelope can contain a document with annotations.  Once the
digital envelope is created, the contents of the envelope can be
posted onto a blockchain and can be accessed by anyone that has the
address of the envelope.  The microservice will be implemented as a
server with an interface using graphql API.  In order to increase
security, our design is such that the keys and authenticating
information not be stored on the server but rather in an external
digital wallet consisting of the TiENChat chat application, and can be
adapted to allow for biometric authentication.

Our microservice provides several blockchain benefits.  First the it
provides timestamp services that insures when the document was created
and the document has not been altered since the time of
creation. Second our system creates a common format that allows
documents to be exchanged on the blockchain and integrated with other
blockchain systemms.  Two companies or industries with different
workflows can perform these workflows internal, but exchange data
using digital envelopes.  Finally, our system allows companies to move
their processes onto blockchain only when needed.

\section{Implementation}

The implementation of our microservice uses as its document store the
Git version control system, and uses the work of the Mango
project\cite{mango} to decentralize the Git version control system \cite{git} to use the
ethereum blockchain and the IPFS filesystem\cite{ipfs} for the data
store.  Our work will require the creation of a GraphQL API to
simplify access in working with documents.  The entire server side
system is encapulated within a docker image making the server system
easily deployable and maintainable with local IT resources in emerging
markets.

The use of the git version control system allows for several
features.  Using the submodule feature, a digital envelope can contain
other digital envelopes.

Our system can be configured to distribute documents both on the
public ethereum and IPFS datastore as well as distributing documents
in private blockchains based on ethereum and IPFS.

\begin{comment}
@startuml

package "Docker Image" {
  [GraphJS API]
  [Git Repository]
  [Mango Interface]
}

cloud {
  [Ethereum Blockchain]
}

cloud {
  [IPFS Data Store]
}

[Document] --> [GraphJS API]
[GraphJS API] --> [Git Repository]
[Git Repository] --> [Mango Interface]
[Mango Interface] --> [Ethereum Blockchain]
[Mango Interface] --> [IPFS Data Store]

@enduml
\end{comment}
\begin{figure}
  \includegraphics[width=\linewidth]{document-server.png}
  \caption{Block Diagram of Microservice}
\end{figure}

\subsection{Current status}

We are currently in the process of implementing our design and the
source code for are systems are available via the joequant/tienlabs
repository.  The first phase of our system will involve the
development of systems that will allow the circulation of public
documents such as business registration certificates.  The second
phase will involve introducing encryption to allow circulation of
private documents.  The third phase will integrate the document
exchange system with third party document management systems such as
MayanEDMS or ERP systems such as Odoo.  We also plan to see if our
system can be used for the decentralized distribution of educational
data by interfacing our systems with Learning Management Systems such
as Moodle.

Simultaneous with our development efforts, we are working with
business partners of the TiEN Foundation in order to deploy our
technology and receive feedback as to the applicability of our
solutions.

\section{Conclusion}
Although blockchain is an extremely promising technology, so far it
has not achieved it's potential due to issues that we have found in
the course of our pilot projects.  We are hoping that the microservice
design which we have undertaken will allow blockchain to achieve its
full potential.

\section{Acknowledgements}
We would like to thank Robin Owmega of Wageni Technologies in
assisting us in setting up our coffee trading pilot project.

\printbibliography
\end{document}
