\documentclass[journal]{IEEEtran}
\begin{document}
\title{Peppercoin: A simple asset backed security token}
\author{Joseph C Wang \thanks{TiENChat Foundation}}
\date{\today}

\maketitle

\begin{abstract}

Although there has been great interest in using security tokens to
allow for the tokenizing of assets, as of late 2019, this effort seems
to have stalled.  Based on our experiences in attempting to tokenize
venture capital funds, we have identified several problems which have
limited the promotion of security tokens.  We believe that these
problems can be resolved with standardized legal documentation, and
have produces an example security token to promote the standarization
of legal documentation.

Recent changes in Hong Kong securities regulation create an opt-in
procedure by which a cryptoexchange that wishes to be regulated by the
Hong Kong regulation can do so by trading a security token.  Motivated
by this regulatory framework, we have created a simple security asset
backed token which can be traded by an exchange which is interested in
being regulated.

In addition to this addition motivation, we hope that 
\end{abstract}

\section{Introduction}
In contrast to the boom in initial coin offerings, the market for
security tokens has been extremely subdued.  In understanding way the
market for security tokens has failed to take off in the way that
initial coin offerings, we have had extension conversations with many
different persons in the cryptocurrency industry, and have concluded
that there are severe limitations in the creation of security tokens.

The initial coin offering market was driven by speculation and
interest from new investors who were interested in either outsize
returns or conversely in moving capital across borders.  Because these
speculative investors were interested in outsized returns, the primary
mentality of the investor was typically of a gambler rather than an
investor.  The speculative bubble also created a large margin which
allowed for cost recovery by service providers.



\subsection{Issues of security tokens}

Current security tokens are completely new offerings.  This creates
extra expense due to the novelity of a security token offering.  The
novelty of security tokens makes them less attractive to purchasers of
tokens.

In addition, the high cost of issuing a security token makes it
unsuitable for certain applications.  Fund managers typically operate
with very low margins, and the high cost of issuing security tokens
prevents certain uses.

There are 

\subsection{Securities regulatory regime under Hong Kong law}

In contrast to the regulatory regimes in other regions, the definition
of securities in Hong Kong can be expanded only by legislative actions
or by addition of securities to the list of securities.

Hong Kong law also allows the definition of securities by
administration action by the Financial Secretary.  However, there are
legal limitations on the ability for the Financial Secretary to do
so.  Administrative actions by the Hong Kong government are can be
challenged on the grounds of ``natural justice''.  Under general
principles of Hong Kong administrative law, a person adversely
affected by an adminstrative action most be given an opportunity to
challenge that action.

In addition, there are constitutional limitations on the ability of
the Hong Kong to regulate securities.  Article 8 of the Basic Law
requires that any changes the 

The result of these limitations is that 

\subsection{Status of Peppercoin under securities law}

\section{Design}

\subsection{Binding problem}
One of the basic issues for virtual assets is to bind to the token to
an owmership right.  A typical example of the binding problem exists
if Alice hands the keys to his house or car to Bob.  In most business
transaction simply handing over the keys to an asset does not result
in an actual transfer of ownership.  Typically both the transferer and
the transferee must go through some legal formality in order to ensure
that there has been in fact a transfer of ownership.

The difficulties in the binding problem can be illustrated by the
following scenario.  Alice runs a cryptoexchange, and has tokens
stolen by Mallory.  Mallory then resells the tokens to Bob who was
unaware that the tokens were stolen.  Alice is unable to locate
Mallory and seeks to recover from Bob.

In most forms of ownership, such as a limited partnership, the
ownership was never transferred from Alice to Mallory, and therefore
Bob does not have proper ownership of the tokens, and Alice can
recover the tokens from Bob.  In the event that Alice is unable to
force Bob to return the tokens, Alice can obtain an injunction
preventing Bob from enjoying any of the benefits of holding the
tokens.

However, if the Alice and Bob are beficienaries of a trust, no
ownership has been transfered.  Hence, if Mallory were to steal tokens
from Alice, Alice could bring an equitable action of unjust enrichment
against Mallory, but if those tokens were transfered in good faith to
Bob, Alice would not have the ability to recover tokens from Bob or to
prevent Bob from enjoying the benefits of holding the tokens.


\section{Implementation of peppercoin}

Peppercoin with two levels.  At the top level, the legal structure of
peppercoin consists of a limited parter

\end{document}

