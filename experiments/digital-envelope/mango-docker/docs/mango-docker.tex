\documentclass[journal]{IEEEtran}
\begin{document}
\title{Digital Envelope Microservices for Blockchain Document Exchange}
\author{Joseph~Chen-Yu~Wang,~\IEEEmembership{TiEN Foundation}
\thanks{e-mail: (joe@tienpay.com).}}
\maketitle

DRAFT - 20181122

\begin{abstract}
Blockchain brings with it the promise of increased efficiency by
allowing for decentralized data exchange between unaffliated parties.
However, the use of blockchain has been limited by the ability to
exchange basic business documents.  This paper outlines the design and
implementation of a digital envelope system, by which arbitrary
documents can be added and annotated into the ethereum blockchain.
\end{abstract}


\begin{IEEEkeywords}
Blockchain
\end{IEEEkeywords}

\section{Introduction}
Blockchain and decentralized finance technology brings with it the
promise of increased efficiency in business processes.  In order to
explore the uses of blockchain, we have attempted to apply blockchain
technology to coffee trading along between Hong Kong and Tanzania.
However in attempting to use blockchain technologies, we have found
that existing systems are not suitable for our efforts.  As such we
have designed and a system which forms the business of blockchain
document exchange.

\subsection{Limitations of Current Systems}

Although there has been much interest in blockchain technology, there
has been very few examples of production uses of blockchain for
business systems.  We have been working with small medium enterprises
involved in trading between Hong Kong and east Africa, and have
attempted to use blockchain technology in support of this business and
as such have developed some insight into the limitations of current
systems.

Among the issues that we have found in attempting to use blockchain
for business applications are
\begin{itemize}
 \item System integration problem
 \item Fixed workflows problem
 \item Walled gardens problem
 \item Weakest link problem
\end{itemize}

\paragraph {System integration problem} Businesses typically have existing
applications which complicated business logic.  Moreover these systems
must be constantly updated to incorporate bug fixes and enhancements,
and it is impractical to put the entire system into the blockchain.

\paragraph {Fixed workflows problem} Another issue is that because blockchain
smart contracts cannot easily be changed after the contract is pushed
to the blockchain, smart contracts typically assume fixed workflows.
In actual business situations, the workflow is subject to change.
Moreover in situations were there are multiple actors, the entire
workflow may not be known to any particular party.

\paragraph {Walled garden issues problem} The usefulness of blockchain is in
the ability to exchanged data between decentralized actors.  However,
most of the blockchain projects that have been proposed thus far
assume that the actors in the blockchain are from similar industries
within a consortium.  However, this prevents blockchain from being
used to exchange information between different industries.

\paragraph {Weakest link issues problem} One further problem with blockchain
systems is that they typically have weakest leak issues in that there
are no productivity improvements when blockchain is used only for part
of the workflow.

Because of these issues our previous efforts to apply blockchain
technology as a business process have been unsuccessful.  However,
our earlier failures in applying blockchain to coffee supply chains
have given us experience for our current effort.

\section{System Design}

One key effort has been to focus on the system design of our system.
Rather than creating a general end to end system which runs into the
various problems, we are creating a microservice that provides simple
document processing functionality which can be incorporated in various
business processes.  In addition to the microservice, we are applying
our software to two specific workflows which can then be used in
actual business application.

Our microservice implements the concept of a digital envelope by which
documents can be placed on the blockchain, and retrieved by users with
the address of the digital envelope.

\section{Implementation}

Implemented as docker imagine using git as document version control
and uses mango distributed git system.  

\section{Conclusion}
We have implemented this system using TiENChat technologies which is
available on github with the directory joequant/tienlabs.  Our
software is issued under the an LGPL license.



% (used to reserve space for the reference number labels box)
\begin{thebibliography}{1}

\bibitem{IEEEhowto:kopka}
H.~Kopka and P.~W. Daly, \emph{A Guide to \LaTeX}, 3rd~ed.\hskip 1em plus
  0.5em minus 0.4em\relax Harlow, England: Addison-Wesley, 1999.

\end{thebibliography}

\end{document}


