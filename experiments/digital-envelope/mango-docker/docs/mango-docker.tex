\documentclass[journal]{IEEEtran}
\begin{document}
\title{Digital Envelopes for Blockchain Document Exchange}
\author{Joseph~Chen-Yu~Wang,~\IEEEmembership{TiEN Foundation}
\thanks{e-mail: (joequant@tienpay.com).}}
\maketitle

\begin{abstract}
Blockchain brings with it the promise of increased efficiency by
allowing for decentralized data exchange between unaffliated parties.
However, the use of blockchain has been limited by the ability to
exchange basic business documents.  This paper outlines the design and
implementation of a digital envelope system, by which arbitrary
documents can be added and annotated into the ethereum blockchain.
\end{abstract}


\begin{IEEEkeywords}
Blockchain
\end{IEEEkeywords}

\section{Introduction}
Blockchain and decentralized finance technology brings with it the
promise of increased efficiency in business processes.  In order to
explore the uses of blockchain, we have attempted to apply blockchain
technology to coffee trading along between Hong Kong and Tanzania.
However in attempting to use blockchain technologies, we have found
that existing systems are not suitable for our efforts.  As such we
have designed and a system which forms the business of blockchain
document exchange.

\subsection{Limitations of Current Systems}

Current blockchain systems are typically based around an enterprise
business process optimization paradigm, in which, there is a fixed
business workflow, and the goal of the blockchain system is to
optimize the business processes.  Another paradigm used is a
consortium private blockchain model by which blockchain systems are
agreed to by a consortium of businesses within a certain industry.

These approaches have a number of limitations.

Actual supply chains and business processes are typically extremely
fluid.  In many situations, the supply chain consists of a number of
uncoordinated actors, and the entire supply chain process is not know
to any particular actor.  

\section{Requirements}

Reduce system to minimial requirements.  Business processes can be
reduced to exchange of documents.

\section{Implementation}

Implemented as docker imagine using git as document control, and used
mango distributed system.

\section{Conclusion}
We have implemented this system using TiENChat technologies.
% (used to reserve space for the reference number labels box)
\begin{thebibliography}{1}

\bibitem{IEEEhowto:kopka}
H.~Kopka and P.~W. Daly, \emph{A Guide to \LaTeX}, 3rd~ed.\hskip 1em plus
  0.5em minus 0.4em\relax Harlow, England: Addison-Wesley, 1999.

\end{thebibliography}

\end{document}


